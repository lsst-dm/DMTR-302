% generated from JIRA project LVV
% using template at /usr/share/miniconda/envs/docsteady-env/lib/python3.7/site-packages/docsteady/templates/tpr.latex.jinja2.
% using docsteady version 2.1.4
% Please do not edit -- update information in Jira instead
\documentclass[DM,lsst,STR,toc]{lsstdoc}
\usepackage{geometry}
\usepackage{longtable,booktabs}
\usepackage{enumitem}
\usepackage{arydshln}
\usepackage{attachfile}
\usepackage{array}
\usepackage{dashrule}

\newcolumntype{L}[1]{>{\raggedright\let\newline\\\arraybackslash\hspace{0pt}}p{#1}}

\input meta.tex

\newcommand{\attachmentsUrl}{https://github.com/\gitorg/\lsstDocType-\lsstDocNum/blob/\gitref/attachments}
\providecommand{\tightlist}{
  \setlength{\itemsep}{0pt}\setlength{\parskip}{0pt}}

\setcounter{tocdepth}{4}

\begin{document}

\def\milestoneName{Science Pipelines Fall 2020 Release}
\def\milestoneId{LDM-503-13a}
\def\product{Science Pipelines SW}

\setDocCompact{true}

\title{LDM-503-13a: Science Pipelines Fall 2020 Release Test Plan and Report}
\setDocRef{\lsstDocType-\lsstDocNum}
\date{ 2021-04-30 }
\author{ Yusra AlSayyad }

% Most recent last
\setDocChangeRecord{
\addtohist{}{2021-04-08}{First draft}{Yusra AlSayyad}
}

\setDocCurator{Yusra AlSayyad}
\setDocUpstreamLocation{\url{https://github.com/lsst-dm/\lsstDocType-\lsstDocNum}}
\setDocUpstreamVersion{\vcsRevision}



\setDocAbstract{
This is the test plan and report for
\textbf{ Science Pipelines Fall 2020 Release} (LDM-503-13a),
an LSST milestone pertaining to the Data Management Subsystem.\\
This document is based on content automatically extracted from the Jira test database on \docDate.
The most recent change to the document repository was on \vcsDate.
}


\maketitle

\section{Introduction}
\label{sect:intro}


\subsection{Objectives}
\label{sect:objectives}

 This test plan checks for the successful release of the Fall 2020
release of the LSST Science Pipelines version 21.0.0\\[2\baselineskip]It
will demonstrate that:\\

\begin{itemize}
\tightlist
\item
  The release has been tagged, built and made available through standard
  distribution channels;
\item
  Release documentation, including release notes and a characterization
  report, are available on the Pipelines documentation website
  (https://pipelines.lsst.io/);
\item
  An end-user can follow standard instructions to install the release
  onto some representative system;
\item
  The release is installed into the ``shared stack'' on the lsst-devl
  shared developer systems at the Rubin Data Facility;
\item
  The pipelines\_check test package executes successfully in the context
  of the release.
\end{itemize}

This test plan does not, in itself, verify the scientific integrity or
algorithmic correctness of the release, beyond checking that defined
procedures for checking basic correctness and characterizing algorithmic
performance have been followed.\\[2\baselineskip]

\subsection{Scope}\label{scope}

The overall strategy for testing and verification within Rubin Data
Management is described in \citeds{LDM-503}.\\
This test plan specifically verifies successful completion of milestone
LDM-503-13a, which refers to the Fall 2020 release of the LSST Science
Pipelines.



\subsection{System Overview}
\label{sect:systemoverview}

 The LSST Science Pipelines comprise the scientific algorithms that will
be used to process data for the Rubin Observatory Legacy Survey of Space
and Time (Rubin's LSST), ~arranged into executable pipelines by means of
the LSST ``task'' framework. They also include execution middleware that
is common across execution environments (for example, the ``Data
Butler'' I/O abstraction is included, but schedulers or workflow
management for specific clusters is not), and ``camera packages'', which
adapt and configure the algorithms for use with specific
instrumentation.\\[2\baselineskip]

\subsection{\texorpdfstring{Applicable Documents\\
}{Applicable Documents }}\label{applicable-documents}

\citeds{LDM-503} Data Management Test Plan\\
\citeds{LDM-151} Data Management Science Pipelines Design\\
\citeds{LSE-61} Data Management System Requirements


\subsection{Document Overview}
\label{sect:docoverview}

This document was generated from Jira, obtaining the relevant information from the
\href{https://jira.lsstcorp.org/secure/Tests.jspa\#/testPlan/LVV-P82}{LVV-P82}
~Jira Test Plan and related Test Cycles (
\href{https://jira.lsstcorp.org/secure/Tests.jspa\#/testCycle/LVV-C171}{LVV-C171}
).

Section \ref{sect:intro} provides an overview of the test campaign, the system under test (\product{}),
the applicable documentation, and explains how this document is organized.
Section \ref{sect:testplan} provides additional information about the test plan, like for example the configuration
used for this test or related documentation.
Section \ref{sect:personnel} describes the necessary roles and lists the individuals assigned to them.

Section \ref{sect:overview} provides a summary of the test results, including an overview in Table \ref{table:summary},
an overall assessment statement and suggestions for possible improvements.
Section \ref{sect:detailedtestresults} provides detailed results for each step in each test case.

The current status of test plan \href{https://jira.lsstcorp.org/secure/Tests.jspa\#/testPlan/LVV-P82}{LVV-P82} in Jira is \textbf{ Approved }.

\subsection{References}
\label{sect:references}
\renewcommand{\refname}{}
\bibliography{lsst,refs,books,refs_ads,local}


\newpage
\section{Test Plan Details}
\label{sect:testplan}


\subsection{Data Collection}

  Observing is not required for this test campaign.

\subsection{Verification Environment}
\label{sect:hwconf}
  Several of the tests described in this plan are agnostic of environment:
they involve checking that certain content has been properly published.
This can be performed from any internet-connected system with a web
browser, and will, in this case, likely be executed from the tester's
laptop.\\[2\baselineskip]Where tests require installation or execution
of specific Science Pipelines components, this will be carried out on
the ``lsst-devl'' shared developer infrastructure at the Rubin Data
Facility. This infrastructure provides a number of powerful (24core,
256GB RAM) systems accessible to Rubin developers. At time of writing,
they are running CentOS 7.9.2009; in practice, any version of CentOS (or
a similar operating system) is appropriate for this test plan, as long
as it complies with the published installation prerequisites of the LSST
Science Pipelines.




\subsection{Related Documentation}


No additional documentation provided.


\subsection{PMCS Activity}

Primavera milestones related to the test campaign:
\begin{itemize}
\item LDM-503-13a
\end{itemize}


\newpage
\section{Personnel}
\label{sect:personnel}

The personnel involved in the test campaign is shown in the following table.

{\small
\begin{longtable}{p{3cm}p{3cm}p{3cm}p{6cm}}
\hline
\multicolumn{2}{r}{T. Plan \href{https://jira.lsstcorp.org/secure/Tests.jspa\#/testPlan/LVV-P82}{LVV-P82} owner:} &
\multicolumn{2}{l}{\textbf{ Yusra AlSayyad } }\\\hline
\multicolumn{2}{r}{T. Cycle \href{https://jira.lsstcorp.org/secure/Tests.jspa\#/testCycle/LVV-C171}{LVV-C171} owner:} &
\multicolumn{2}{l}{\textbf{
Yusra AlSayyad }
} \\\hline
\textbf{Test Cases} & \textbf{Assigned to} & \textbf{Executed by} & \textbf{Additional Test Personnel} \\ \hline
\href{https://jira.lsstcorp.org/secure/Tests.jspa#/testCase/LVV-T362}{LVV-T362}
& {\small Yusra AlSayyad } & {\small Yusra AlSayyad } &
\begin{minipage}[]{6cm}
\smallskip
{\small  }
\medskip
\end{minipage}
\\ \hline
\href{https://jira.lsstcorp.org/secure/Tests.jspa#/testCase/LVV-T1601}{LVV-T1601}
& {\small Yusra AlSayyad } & {\small Yusra AlSayyad } &
\begin{minipage}[]{6cm}
\smallskip
{\small  }
\medskip
\end{minipage}
\\ \hline
\href{https://jira.lsstcorp.org/secure/Tests.jspa#/testCase/LVV-T363}{LVV-T363}
& {\small Yusra AlSayyad } & {\small Yusra AlSayyad } &
\begin{minipage}[]{6cm}
\smallskip
{\small  }
\medskip
\end{minipage}
\\ \hline
\end{longtable}
}

\newpage

\section{Test Campaign Overview}
\label{sect:overview}

\subsection{Summary}
\label{sect:summarytable}

{\small
\begin{longtable}{p{2cm}cp{2.3cm}p{8.6cm}p{2.3cm}}
\toprule
\multicolumn{2}{r}{ T. Plan \href{https://jira.lsstcorp.org/secure/Tests.jspa\#/testPlan/LVV-P82}{LVV-P82}:} &
\multicolumn{2}{p{10.9cm}}{\textbf{ LDM-503-13a: Science Pipelines Fall 2020 Release }} & Approved \\\hline
\multicolumn{2}{r}{ T. Cycle \href{https://jira.lsstcorp.org/secure/Tests.jspa\#/testCycle/LVV-C171}{LVV-C171}:} &
\multicolumn{2}{p{10.9cm}}{\textbf{ LDM-503-13a: Science Pipelines Fall 2020 Release }} & Done \\\hline
\textbf{Test Cases} &  \textbf{Ver.} & \textbf{Status} & \textbf{Comment} & \textbf{Issues} \\\toprule
\href{https://jira.lsstcorp.org/secure/Tests.jspa#/testCase/LVV-T362}{LVV-T362}
&  1
& Pass w/ Deviation &
\begin{minipage}[]{9cm}
\smallskip
Test completed with deviation; see linked issue
\medskip
\end{minipage}
&   \href{https://jira.lsstcorp.org/browse/DM-29904}{DM-29904}
\\\hline
\href{https://jira.lsstcorp.org/secure/Tests.jspa#/testCase/LVV-T1601}{LVV-T1601}
&  1
& Pass w/ Deviation &
\begin{minipage}[]{9cm}
\smallskip
Test completed with deviation
\medskip
\end{minipage}
&   \href{https://jira.lsstcorp.org/browse/DM-28988}{DM-28988}
\href{https://jira.lsstcorp.org/browse/DM-29963}{DM-29963}
\\\hline
\href{https://jira.lsstcorp.org/secure/Tests.jspa#/testCase/LVV-T363}{LVV-T363}
&  1
& Pass &
\begin{minipage}[]{9cm}
\smallskip
Test is completed
\medskip
\end{minipage}
&   \\\hline
\caption{Test Campaign Summary}
\label{table:summary}
\end{longtable}
}

\subsection{Overall Assessment}
\label{sect:overallassessment}

All the tests described in this plan were completed
successfully.\\[2\baselineskip]Two minor deviations were encountered.\\
(1) This release marks the full replacement of the lsst\_dm\_stack\_demo
package with the pipelines\_check package. The instructions on the
pipelines\_check README were missing a step if followed directly from
the stack installation procedure in the previous test case step. This
was resolved on DM-29904.\\
(2) The shared stack maintained for developers at the Data Facility
included a tagged version for the v21 release candidate, it did not
contain a tagged version for the final v21 release. This distinction is
minor because the two versions are equivalent, but this was resolved on
DM-29963.

\subsection{Recommended Improvements}
\label{sect:recommendations}

No recommendations beyond the tickets filed

\newpage
\section{Detailed Test Results}
\label{sect:detailedtestresults}

\subsection{Test Cycle LVV-C171 }

Open test cycle {\it \href{https://jira.lsstcorp.org/secure/Tests.jspa#/testrun/LVV-C171}{LDM-503-13a: Science Pipelines Fall 2020 Release}} in Jira.

Test Cycle name: LDM-503-13a: Science Pipelines Fall 2020 Release\\
Status: Done

This test cycle describes tests performed on the Science Pipelines Fall
2020 (v21.0.0) release, ensuring that the release is properly
identified, documented, distributed, installable and tested.

\subsubsection{Software Version/Baseline}
A web browser is required for inspecting release artifacts published to
the web.\\[2\baselineskip]Testing the software installation procedures,
and demonstrating that the release's integration tests can be executed
successfully, require a supported operating system with the documented
prerequisites of the release installed. This will be carried out on the
''lsst-dev'' shared developer systems at the LSST Data
Facility.\\[2\baselineskip]At time of writing, Data Facility systems run
CentOS Linux release 7.9.2009, and it is anticipated that this will be a
supported platform for the Science Pipelines
release.\\[2\baselineskip]Science Pipelines prerequisites are documented
at \href{https://pipelines.lsst.io/}{pipelines.lsst.io}; all of these
must be installed.\\[2\baselineskip]It is possible that the software
release will involve a reorganization of documentation describing
prerequisites; in this case, the documentation corresponding to the new
release must be consulted.

\subsubsection{Configuration}
No specific configuration is required beyond the availability of test
systems with the prerequisite software, described above, installed.

\subsubsection{Test Cases in LVV-C171 Test Cycle}

\paragraph{ LVV-T362 - Installation of the LSST Science Pipelines Payloads }\mbox{}\\

Version \textbf{1}.
Open  \href{https://jira.lsstcorp.org/secure/Tests.jspa#/testCase/LVV-T362}{\textit{ LVV-T362 } }
test case in Jira.

This test will check that:

\begin{itemize}
\tightlist
\item
  The Alert Production Pipeline payload is available for installation
  from documented channels;
\item
  The Data Release Production Pipeline payload is available for
  installation from documented channels;
\item
  The Calibration Products Production Pipeline payload is available for
  installation from documented channels;
\item
  These payloads can be installed on systems at the LSST Data Facility
  following available documentation;
\item
  The installed pipeline payloads are capable of successfully executing
  basic integration tests.
\end{itemize}

Note that this test assumes a 2018-era packaging of the Science
Pipelines software, in which all the above payloads are represented by a
single ``meta-package'', lsst\_distrib.

\textbf{ Preconditions}:\\


Execution status: {\bf Pass w/ Deviation }

Final comment:\\Test completed with deviation; see linked issue

Issues found during the execution of LVV-T362 test case:

\begin{itemize}
\item \href{https://jira.lsstcorp.org/browse/DM-29904}{DM-29904}~~Instruction on pipelines\_check README are incomplete

\end{itemize}

Detailed steps results:

\begin{tabular}{p{2cm}p{14cm}}
\toprule
Step 1 & Step Execution Status: \textbf{ Pass } \\ \hline
\end{tabular}
 Description \\
{\footnotesize
The LSST Science Pipelines, described by the lsst\_distrib meta-package,
should be installed following the documentation available at
https://pipelines.lsst.io/. The suggested Conda environment will be used
to ensure that a supported execution environment is available.

}
\hdashrule[0.5ex]{\textwidth}{1pt}{3mm}
  Expected Result \\
{\footnotesize
Detailed output will depend on the installation method chosen, but will
confirm the successful installation of the Science Pipelines.

}
\hdashrule[0.5ex]{\textwidth}{1pt}{3mm}
  Actual Result \\
{\footnotesize
The Science Pipelines were installed on the shared server
lsst-devl01.ncsa.illinois.edu which is running CentOS 7.9.2009, using
the procedure described in
https://pipelines.lsst.io/install/newinstall.html\\[2\baselineskip]Following
the instructions in step 2 and 3, I made an installation directory and
ran newinstall.sh:\\

\begin{verbatim}
mkdir -p lsst_stack
cd lsst_stack
curl -OL https://raw.githubusercontent.com/lsst/lsst/21.0.0/scripts/newinstall.sh
bash newinstall.sh -ct
\end{verbatim}

answering `'yes'' when prompted to install
miniconda:\\[2\baselineskip]\textgreater{} Would you like us to install
Miniconda distribution (if unsure, say yes)?yes\\[2\baselineskip]Then
following step 4, I installed the Science Pipelines:\\

\begin{verbatim}
source loadLSST.bash # for bash
eups distrib install -t v21_0_0 lsst_distrib
\end{verbatim}

\begin{verbatim}
curl -sSL https://raw.githubusercontent.com/lsst/shebangtron/master/shebangtron | python
setup lsst_distrib
\end{verbatim}

}
\begin{tabular}{p{2cm}p{14cm}}
\toprule
Step 2 & Step Execution Status: \textbf{ Pass } \\ \hline
\end{tabular}
 Description \\
{\footnotesize
The lsst\_distrib top-level metapackage will be enabled. Assuming that
the software has been installed at
\$\{LSST\_DIR\}:\\[2\baselineskip]\hspace*{0.333em} ~ ~ ~source
\$\{LSST\_DIR\}/loadLSST.bash\\
\hspace*{0.333em} ~ ~ ~setup lsst\_distrib

}
\hdashrule[0.5ex]{\textwidth}{1pt}{3mm}
  Expected Result \\
{\footnotesize
Nothing is printed. The command\\[2\baselineskip]\hspace*{0.333em} ~eups
list -s lsst\_distrib\\[2\baselineskip]may be used to confirm that the
correct version of the codebase has been installed.

}
\hdashrule[0.5ex]{\textwidth}{1pt}{3mm}
  Actual Result \\
{\footnotesize
(lsst-scipipe-cb4e2dc) {[}yusra@lsst-devl01 lsst\_stack{]}\$ source
/project/yusra/LVV-C171/TVV-T362/lsst\_stack/loadLSST.bash\\
(lsst-scipipe-cb4e2dc) {[}yusra@lsst-devl01 lsst\_stack{]}\$ setup
lsst\_distrib\\
(lsst-scipipe-cb4e2dc) {[}yusra@lsst-devl01 lsst\_stack{]}\$ eups list
-s lsst\_distrib\\
21.0.0+973e4c9e85 current v21\_0\_0 setup\\[2\baselineskip]

}
\begin{tabular}{p{2cm}p{14cm}}
\toprule
Step 3 & Step Execution Status: \textbf{ Pass } \\ \hline
\end{tabular}
 Description \\
{\footnotesize
The ``LSST Stack Demo'' package will be downloaded onto the test system
from https://github.com/lsst/pipelines\_check/releases. The version
corresponding to to the version of the Science Pipelines under test
should be chosen.

}
\hdashrule[0.5ex]{\textwidth}{1pt}{3mm}
  Expected Result \\
{\footnotesize
Depends on the tool selected by the user for downloading.

}
\hdashrule[0.5ex]{\textwidth}{1pt}{3mm}
  Actual Result \\
{\footnotesize
The 21.0.0 pipelines\_check was downloaded from
https://github.com/lsst/pipelines\_check/releases/tag/21.0.0 by
executing the command:\\[2\baselineskip]

\begin{verbatim}
(lsst-scipipe-cb4e2dc) [yusra@lsst-devl01]$ curl -OL https://github.com/lsst/pipelines_check/archive/refs/tags/21.0.0.tar.gz
% Total % Received % Xferd Average Speed Time Time Time Current
Dload Upload Total Spent Left Speed
100 138 100 138 0 0 862 0 --:--:-- --:--:-- --:--:-- 862
100 25.8M 0 25.8M 0 0 13.6M 0 --:--:-- 0:00:01 --:--:-- 16.8M
\end{verbatim}

}
\begin{tabular}{p{2cm}p{14cm}}
\toprule
Step 4 & Step Execution Status: \textbf{ Pass } \\ \hline
\end{tabular}
 Description \\
{\footnotesize
The stack demo package is uncompressed into a directory \$\{DEMO\_DIR\}.

}
\hdashrule[0.5ex]{\textwidth}{1pt}{3mm}
  Expected Result \\
{\footnotesize
Depends on options given to the tar command. Should confirm the
availability of the stack demo source.

}
\hdashrule[0.5ex]{\textwidth}{1pt}{3mm}
  Actual Result \\
{\footnotesize
\begin{verbatim}
The demo package was uncompressed by running:

$ tar -xzf 21.0.0.tar.gz
\end{verbatim}

There is now an uncompressed directory called pipelines\_check-21.0.0.

}
\begin{tabular}{p{2cm}p{14cm}}
\toprule
Step 5 & Step Execution Status: \textbf{ Pass w/ Deviation } \\ \hline
\end{tabular}
 Description \\
{\footnotesize
The demo package will be executed by following the instructions in its
README file.~

}
\hdashrule[0.5ex]{\textwidth}{1pt}{3mm}
  Expected Result \\
{\footnotesize
Successful execution will result in the string ``Ok'' being returned.

}
\hdashrule[0.5ex]{\textwidth}{1pt}{3mm}
  Actual Result \\
{\footnotesize
The pipeline\_check completed successfully with a minor deviation of the
instructions on the package README. ~The instructions for execution were
to\\
\texttt{\$\ source\ \$LSST\_HOME/loadLSST.sh\$\ setup\ -r\ .\$\ ./bin/run\_demo.sh}\\
loadLSST is spelled loadLSST.bash for bash, and the `setup' command
requires either the `-j' argument or an extra stack installation step of
cloning the v21\_0\_0 tag as current first:\\

\begin{verbatim}
(lsst-scipipe-cb4e2dc) [yusra@lsst-devl01 pipelines_check-21.0.0]$ setup -r .
setup: in file /project/yusra/LVV-C171/TVV-T362/pipelines_check-21.0.0/ups/pipelines_check.table: Product pipe_tasks not found
(lsst-scipipe-cb4e2dc) [yusra@lsst-devl01 pipelines_check-21.0.0]$ eups list -s pipe_tasks
21.0.0+44ca056b81 v21_0_0 setup
(lsst-scipipe-cb4e2dc) [yusra@lsst-devl01 pipelines_check-21.0.0]$ eups tags --clone v21_0_0 current
(lsst-scipipe-cb4e2dc) [yusra@lsst-devl01 pipelines_check-21.0.0]$ setup -r .
(lsst-scipipe-cb4e2dc) [yusra@lsst-devl01 pipelines_check-21.0.0]$ ./bin/run_demo.sh
+ '[' '!' -f DATA_REPO/butler.yaml ']'
+ butler create DATA_REPO
+ butler register-instrument DATA_REPO lsst.obs.subaru.HyperSuprimeCam
+ '[' '!' -d DATA_REPO/HSC/calib ']'
+ butler import DATA_REPO /project/yusra/LVV-C171/TVV-T362/pipelines_check-21.0.0/input_data --export-file /project/yusra/LVV-C171/TVV-T362/pipelines_check-21.0.0/input_data/export.yaml --skip-dimensions instrument,physical_filter,detector

<log snipped>

INFO 2021-04-26T23:02:25.852-0500 calibrate (calibrate:{band: r, instrument: HSC, detector: 10, physical_filter: HSC-R, visit_system: 0, visit: 903342})(calibrate.py:748)- Photometric zero-point: 30.549066
+ pipetask qgraph -b DATA_REPO/butler.yaml --input HSC/calib,HSC/raw/all,refcats -p /project/yusra/LVV-C171/TVV-T362/lsst_stack/stack/miniconda3-py37_4.8.2-cb4e2dc/Linux64/pipe_tasks/21.0.0+44ca056b81/pipelines/DRP.yaml:processCcd --instrument lsst.obs.subaru.HyperSuprimeCam --output-run demo_collection_1
ctrl.mpexec.cmdLineFwk INFO: QuantumGraph contains 3 quanta for 3 tasks
(lsst-scipipe-cb4e2dc) [yusra@lsst-devl01 pipelines_check-21.0.0]$
\end{verbatim}

}

\paragraph{ LVV-T1601 - Science Pipelines available on developer hardware }\mbox{}\\

Version \textbf{1}.
Open  \href{https://jira.lsstcorp.org/secure/Tests.jspa#/testCase/LVV-T1601}{\textit{ LVV-T1601 } }
test case in Jira.

This test will check that a given release of the LSST Science Pipelines
is available for use in a ``shared stack'' on developer infrastructure.

\textbf{ Preconditions}:\\


Execution status: {\bf Pass w/ Deviation }

Final comment:\\Test completed with deviation

Issues found during the execution of LVV-T1601 test case:

\begin{itemize}
\item \href{https://jira.lsstcorp.org/browse/DM-28988}{DM-28988}~~Update https://developer.lsst.io/services/software.html\#software-stack

\item \href{https://jira.lsstcorp.org/browse/DM-29963}{DM-29963}~~Build version 21.0.0 in the shared stack

\end{itemize}

Detailed steps results:

\begin{tabular}{p{2cm}p{14cm}}
\toprule
Step 1 & Step Execution Status: \textbf{ Pass } \\ \hline
\end{tabular}
 Description \\
{\footnotesize
Consult the LSST Developer Guide (http://developer.lsst.io/) to
establish:\\[2\baselineskip]

\begin{itemize}
\tightlist
\item
  An appropriate hostname and login instructions for the shared
  developer infrastructure at the LSST Data Facility;
\item
  Instructions for initializing the shared stack on the developer
  infrastructure.
\end{itemize}

}
\hdashrule[0.5ex]{\textwidth}{1pt}{3mm}
  Expected Result \\
{\footnotesize
The Developer Guide clearly presents information about connecting to and
using shared infrastructure.

}
\hdashrule[0.5ex]{\textwidth}{1pt}{3mm}
  Actual Result \\
{\footnotesize
\begin{itemize}
\tightlist
\item
  Instructions for logging into the head nodes lsst-devl0* at the NCSA
  development cluster are available at
  https://developer.lsst.io/services/lsst-login.html
\item
  Instructions for accessing the shared stack are available at
  https://developer.lsst.io/services/software.html\#software-stack
\end{itemize}

}
\begin{tabular}{p{2cm}p{14cm}}
\toprule
Step 2 & Step Execution Status: \textbf{ Pass } \\ \hline
\end{tabular}
 Description \\
{\footnotesize
Connect to the shared infrastructure following the Developer Guide
instructions.

}
\hdashrule[0.5ex]{\textwidth}{1pt}{3mm}
  Expected Result \\
{\footnotesize
A shell prompt on a shared machine.

}
\hdashrule[0.5ex]{\textwidth}{1pt}{3mm}
  Actual Result \\
{\footnotesize
login successful and a shell prompt is available.~

}
\begin{tabular}{p{2cm}p{14cm}}
\toprule
Step 3 & Step Execution Status: \textbf{ Pass } \\ \hline
\end{tabular}
 Description \\
{\footnotesize
Initialize the LSST environment following the Developer Guide
instructions.

}
\hdashrule[0.5ex]{\textwidth}{1pt}{3mm}
  Example Code \\
{\footnotesize
source /software/lsstsw/stack/loadLSST.bash

}
\hdashrule[0.5ex]{\textwidth}{1pt}{3mm}
  Expected Result \\
{\footnotesize
No errors are shown.

}
\hdashrule[0.5ex]{\textwidth}{1pt}{3mm}
  Actual Result \\
{\footnotesize
The shared stack for the v21 release is available at
~/software/lsstsw/stack\_20200922. The command executed was:\\
\texttt{source\ /software/lsstsw/stack\_20200922/loadLSST.bash}

}
\begin{tabular}{p{2cm}p{14cm}}
\toprule
Step 4 & Step Execution Status: \textbf{ Pass w/ Deviation } \\ \hline
\end{tabular}
 Description \\
{\footnotesize
List available software products using EUPS, and check that the release
under test is available.

}
\hdashrule[0.5ex]{\textwidth}{1pt}{3mm}
  Example Code \\
{\footnotesize
eups list lsst\_distrib \textbar{} grep vXX\_XX\_XX ~\# replace
XX\_XX\_XX with the version number of the release under test

}
\hdashrule[0.5ex]{\textwidth}{1pt}{3mm}
  Expected Result \\
{\footnotesize
The provided version number should be available in the shared stack.

}
\hdashrule[0.5ex]{\textwidth}{1pt}{3mm}
  Actual Result \\
{\footnotesize
There is no v21\_0\_0 tag in the shared stack, however developers have
been using the tag for release candidate v21\_0\_0\_rc1 instead when a
v21 stack is needed. ~v21\_0\_0\_rc1 became v21\_0\_0, and they are
equivalent:\\

\begin{verbatim}
(lsst-scipipe) [yusra@lsst-devl01]$ eups list lsst_distrib | grep v21_0_0
20.0.0-3-gdc8c9d3+973e4c9e85 v21_0_0_rc1
\end{verbatim}

}

\paragraph{ LVV-T363 - Science Pipelines Release Documentation }\mbox{}\\

Version \textbf{1}.
Open  \href{https://jira.lsstcorp.org/secure/Tests.jspa#/testCase/LVV-T363}{\textit{ LVV-T363 } }
test case in Jira.

This test will check:

\begin{itemize}
\tightlist
\item
  That a particular Science Pipelines release is adequately described by
  documentation at the https://pipelines.lsst.io/ site;
\item
  That the Science Pipelines release is accompanied by a
  characterization report which describes its scientific performance.
\end{itemize}

\textbf{ Preconditions}:\\


Execution status: {\bf Pass }

Final comment:\\Test is completed


Detailed steps results:

\begin{tabular}{p{2cm}p{14cm}}
\toprule
Step 1 & Step Execution Status: \textbf{ Pass } \\ \hline
\end{tabular}
 Description \\
{\footnotesize
Load the Science Pipelines website at https://pipelines.lsst.io/.

}
\hdashrule[0.5ex]{\textwidth}{1pt}{3mm}
  Expected Result \\
{\footnotesize
The website is displayed.

}
\hdashrule[0.5ex]{\textwidth}{1pt}{3mm}
  Actual Result \\
{\footnotesize
The website https://pipelines.lsst.io is displayed~

}
\begin{tabular}{p{2cm}p{14cm}}
\toprule
Step 2 & Step Execution Status: \textbf{ Pass } \\ \hline
\end{tabular}
 Description \\
{\footnotesize
Identify documentation for the release under test. This should be
clearly labelled on the documentation site.\\[2\baselineskip]If the
latest release is being tested, the default page loaded when visiting
https://pipelines.lsst.io/ should be the documentation
required.\\[2\baselineskip]If this test is for another release, the site
should present clear instructions for changing the edition (or version)
of the documentation being examined, and documentation for the release
under test should be available.

}
\hdashrule[0.5ex]{\textwidth}{1pt}{3mm}
  Expected Result \\
{\footnotesize
The documentation for the release under test is displayed.

}
\hdashrule[0.5ex]{\textwidth}{1pt}{3mm}
  Actual Result \\
{\footnotesize
https://pipelines.lsst.io is for the release under test and clearly
states: ''This documentation covers version \textbf{v21\_0\_0''~}\\
It was also simple to navigate to ``Other Versions'', and click on
v21\_0\_0 and land on the permanent page address for this release:
https://pipelines.lsst.io/v/v21\_0\_0/index.html\\[2\baselineskip]

}
\begin{tabular}{p{2cm}p{14cm}}
\toprule
Step 3 & Step Execution Status: \textbf{ Pass } \\ \hline
\end{tabular}
 Description \\
{\footnotesize
Inspect the documentation to ensure that it refers to the release under
test, and that it provides:

\begin{itemize}
\tightlist
\item
  Release notes, describing changes in this release relative to the
  previous;
\item
  Installation instructions, together with a list of supported platforms
  and prerequisites;
\item
  Getting started information.
\end{itemize}

}
\hdashrule[0.5ex]{\textwidth}{1pt}{3mm}
  Expected Result \\
{\footnotesize
The user is satisfied that the required information is available.

}
\hdashrule[0.5ex]{\textwidth}{1pt}{3mm}
  Actual Result \\
{\footnotesize
The documentation for the following are all available on the front page
or linked from the front page:

\begin{itemize}
\tightlist
\item
  Release notes:
  https://pipelines.lsst.io/v/v21\_0\_0/releases/v21\_0\_0.html\#release-latest
\item
  Installation instructions:
  https://pipelines.lsst.io/v/v21\_0\_0/index.html\#installation
\item
  Getting started information:
  https://pipelines.lsst.io/v/v21\_0\_0/index.html\#getting-started
\end{itemize}

}
\begin{tabular}{p{2cm}p{14cm}}
\toprule
Step 4 & Step Execution Status: \textbf{ Pass } \\ \hline
\end{tabular}
 Description \\
{\footnotesize
Locate the Characterization Metric Report corresponding to this release.
It should be linked from the main release documentation.

}
\hdashrule[0.5ex]{\textwidth}{1pt}{3mm}
  Expected Result \\
{\footnotesize
The user is satisfied that the report is available.

}
\hdashrule[0.5ex]{\textwidth}{1pt}{3mm}
  Actual Result \\
{\footnotesize
The Characterization Report for this release
\href{https://ls.st/DMTR-281}{Characterization Metric Report
(DMTR-281)}~ is clearly linked from the Release Notes:
~https://pipelines.lsst.io/v/v21\_0\_0/releases/v21\_0\_0.html\#release-latest

}
\begin{tabular}{p{2cm}p{14cm}}
\toprule
Step 5 & Step Execution Status: \textbf{ Pass } \\ \hline
\end{tabular}
 Description \\
{\footnotesize
Verify that the characterization metric report describes the scientific
performance of the release in terms of a selection of performance
metrics drawn from high-level requirements documentation (the Science
Requirements Document, LPM-17; the LSST System Requirements, LSE-29;
and/or the Observatory System Specifications, LSE-30).

}
\hdashrule[0.5ex]{\textwidth}{1pt}{3mm}
  Expected Result \\
{\footnotesize
Metric values describing the performance of the release, for example as
computed by validate\_drp, are described in the report.

}
\hdashrule[0.5ex]{\textwidth}{1pt}{3mm}
  Actual Result \\
{\footnotesize
The Characterization Report includes the expected metrics.~

}




\newpage
\appendix
% Make sure lsst-texmf/bin/generateAcronyms.py is in your path
\section{Acronyms used in this document}\label{sec:acronyms}
\input{acronyms.tex}

\newpage

\input{DMTR-302.appendix.tex}

\end{document}
